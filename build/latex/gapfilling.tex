%% Generated by Sphinx.
\def\sphinxdocclass{report}
\documentclass[letterpaper,10pt,english]{sphinxmanual}
\ifdefined\pdfpxdimen
   \let\sphinxpxdimen\pdfpxdimen\else\newdimen\sphinxpxdimen
\fi \sphinxpxdimen=.75bp\relax

\PassOptionsToPackage{warn}{textcomp}
\usepackage[utf8]{inputenc}
\ifdefined\DeclareUnicodeCharacter
% support both utf8 and utf8x syntaxes
  \ifdefined\DeclareUnicodeCharacterAsOptional
    \def\sphinxDUC#1{\DeclareUnicodeCharacter{"#1}}
  \else
    \let\sphinxDUC\DeclareUnicodeCharacter
  \fi
  \sphinxDUC{00A0}{\nobreakspace}
  \sphinxDUC{2500}{\sphinxunichar{2500}}
  \sphinxDUC{2502}{\sphinxunichar{2502}}
  \sphinxDUC{2514}{\sphinxunichar{2514}}
  \sphinxDUC{251C}{\sphinxunichar{251C}}
  \sphinxDUC{2572}{\textbackslash}
\fi
\usepackage{cmap}
\usepackage[T1]{fontenc}
\usepackage{amsmath,amssymb,amstext}
\usepackage{babel}



\usepackage{times}
\expandafter\ifx\csname T@LGR\endcsname\relax
\else
% LGR was declared as font encoding
  \substitutefont{LGR}{\rmdefault}{cmr}
  \substitutefont{LGR}{\sfdefault}{cmss}
  \substitutefont{LGR}{\ttdefault}{cmtt}
\fi
\expandafter\ifx\csname T@X2\endcsname\relax
  \expandafter\ifx\csname T@T2A\endcsname\relax
  \else
  % T2A was declared as font encoding
    \substitutefont{T2A}{\rmdefault}{cmr}
    \substitutefont{T2A}{\sfdefault}{cmss}
    \substitutefont{T2A}{\ttdefault}{cmtt}
  \fi
\else
% X2 was declared as font encoding
  \substitutefont{X2}{\rmdefault}{cmr}
  \substitutefont{X2}{\sfdefault}{cmss}
  \substitutefont{X2}{\ttdefault}{cmtt}
\fi


\usepackage[Bjarne]{fncychap}
\usepackage{sphinx}

\fvset{fontsize=\small}
\usepackage{geometry}


% Include hyperref last.
\usepackage{hyperref}
% Fix anchor placement for figures with captions.
\usepackage{hypcap}% it must be loaded after hyperref.
% Set up styles of URL: it should be placed after hyperref.
\urlstyle{same}
\addto\captionsenglish{\renewcommand{\contentsname}{Contents:}}

\usepackage{sphinxmessages}
\setcounter{tocdepth}{1}



\title{Gap Filling Documentation}
\date{Nov 30, 2020}
\release{v0.1}
\author{Bayer}
\newcommand{\sphinxlogo}{\vbox{}}
\renewcommand{\releasename}{Release}
\makeindex
\begin{document}

\pagestyle{empty}
\sphinxmaketitle
\pagestyle{plain}
\sphinxtableofcontents
\pagestyle{normal}
\phantomsection\label{\detokenize{index::doc}}



\chapter{Ordeting Data}
\label{\detokenize{ordering_data:ordeting-data}}\label{\detokenize{ordering_data::doc}}

\section{Ordering data from the EUMETSAT archive}
\label{\detokenize{ordering_data:ordering-data-from-the-eumetsat-archive}}

\subsection{1. Log in to https://eoportal.eumetsat.int with:}
\label{\detokenize{ordering_data:log-in-to-https-eoportal-eumetsat-int-with}}
\sphinxhref{./images/ordering\_data\_1.png}{\sphinxincludegraphics{{ordering_data_1}.png}}

\begin{sphinxVerbatim}[commandchars=\\\{\}]
username: hdeneke
password: S3V1R1umarf
\end{sphinxVerbatim}


\subsection{2. Start the data centre application using \sphinxstylestrong{Access} under \sphinxstylestrong{Data Centre}.}
\label{\detokenize{ordering_data:start-the-data-centre-application-using-access-under-data-centre}}
\sphinxhref{./images/ordering\_data\_2.png}{\sphinxincludegraphics{{ordering_data_2}.png}}


\subsection{3. Select your data type from the data center product list, then go on with \sphinxstylestrong{next step}:}
\label{\detokenize{ordering_data:select-your-data-type-from-the-data-center-product-list-then-go-on-with-next-step}}
\sphinxhref{./images/ordering\_data\_3.png}{\sphinxincludegraphics{{ordering_data_3}.png}}
\begin{itemize}
\item {} 
For \sphinxstylestrong{Prime\sphinxhyphen{}Service} (PZS) use: \sphinxstyleemphasis{“High Rate SEVIRI Level 1.5 Image Data \textendash{} MSG \textendash{} 0 degree”}

\item {} 
For \sphinxstylestrong{Rapid\sphinxhyphen{}Scan\sphinxhyphen{}Service} (RSS) use: \sphinxstyleemphasis{“Rapid Scan High Rate SEVIRI Level 1.5 Image Data \textendash{} MSG”}

\item {} 
For \sphinxstylestrong{Prime\sphinxhyphen{}Service over Indian Ocean} use: \sphinxstyleemphasis{“High Rate SEVIRI Level 1.5 Image Data \textendash{} MSG \textendash{} Indian Ocean 41.5 degrees E”}

\end{itemize}


\subsection{4. Double\sphinxhyphen{}check if the required longitude in \sphinxstylestrong{Sub Sat Longitude} is correct:}
\label{\detokenize{ordering_data:double-check-if-the-required-longitude-in-sub-sat-longitude-is-correct}}
\sphinxhref{./images/ordering\_data\_4.png}{\sphinxincludegraphics{{ordering_data_4}.png}}
\begin{itemize}
\item {} 
0° for the Prime\sphinxhyphen{}Service (PZS)

\item {} 
9.5° for the Rapid\sphinxhyphen{}Scan\sphinxhyphen{}Service (RSS)

\item {} 
41.5° for the Indian Ocean

\end{itemize}


\subsection{5. Select the required period of time in the \sphinxstylestrong{Select Date / Time} field. Get results with \sphinxstylestrong{Apply} (might take a while).}
\label{\detokenize{ordering_data:select-the-required-period-of-time-in-the-select-date-time-field-get-results-with-apply-might-take-a-while}}
\sphinxhref{./images/ordering\_data\_5.png}{\sphinxincludegraphics{{ordering_data_5}.png}}
\begin{itemize}
\item {} 
a maximum of \sphinxstylestrong{three months} for Prime\sphinxhyphen{}Service (PZS)

\item {} 
a maximum of \sphinxstylestrong{one month} for Rapid\sphinxhyphen{}Scan\sphinxhyphen{}Service (RSS)

\end{itemize}


\subsection{6. Skip the selection of regions using \sphinxstylestrong{Next Step}.}
\label{\detokenize{ordering_data:skip-the-selection-of-regions-using-next-step}}
\sphinxhref{./images/ordering\_data\_6.png}{\sphinxincludegraphics{{ordering_data_6}.png}}


\subsection{7. Select product order format as “\sphinxstylestrong{HRIT data sets in tar file}”. Then click next step.}
\label{\detokenize{ordering_data:select-product-order-format-as-hrit-data-sets-in-tar-file-then-click-next-step}}
\sphinxhref{./images/ordering\_data\_7.png}{\sphinxincludegraphics{{ordering_data_7}.png}}


\subsection{8. Go to \sphinxstylestrong{Details}, which on the right to \sphinxstylestrong{next step}.}
\label{\detokenize{ordering_data:go-to-details-which-on-the-right-to-next-step}}
\sphinxhref{./images/ordering\_data\_8.png}{\sphinxincludegraphics{{ordering_data_8}.png}}

Use ◄ and ► to browse through the selected period of time and use \sphinxstylestrong{CTRL + Left Click} to select the slots you want. If you are done with
the selection use \sphinxstylestrong{Remove Unselected} to clear all unwanted slots from your list. Double\sphinxhyphen{}check if no slot is missing. Use “Close” if everything
is correct.


\subsection{9. Choose the \sphinxstylestrong{Delivery method} which is suitable for the data:}
\label{\detokenize{ordering_data:choose-the-delivery-method-which-is-suitable-for-the-data}}
\sphinxhref{./images/ordering\_data\_9.png}{\sphinxincludegraphics{{ordering_data_9}.png}}

\(\rightarrow\) “Online HTTP” if the order is \sphinxstylestrong{bigger than 80 GB}

\(\rightarrow\) “On Media” if the order is \sphinxstylestrong{less than 80 GB}

By delivery option \sphinxstylestrong{no compression} is fine because the data files are already compressed.


\subsection{10. Go the last step, where only the overall data size is presented.}
\label{\detokenize{ordering_data:go-the-last-step-where-only-the-overall-data-size-is-presented}}
\sphinxhref{./images/ordering\_data\_10.png}{\sphinxincludegraphics{{ordering_data_10}.png}}

Finally send the order with “Place your order”.


\subsection{11. Note the order number at our wiki:}
\label{\detokenize{ordering_data:note-the-order-number-at-our-wiki}}
\sphinxurl{http://wiki-intern.tropos.de/index.php/EUMETSAT\_Data\_Ordering\_Diary}

\sphinxhref{./images/ordering\_data\_11.png}{\sphinxincludegraphics{{ordering_data_11}.png}}


\subsection{12. Click at \sphinxstylestrong{ORDER STATUS}. It shows the order number and the status of the order. Several status are possible:}
\label{\detokenize{ordering_data:click-at-order-status-it-shows-the-order-number-and-the-status-of-the-order-several-status-are-possible}}\begin{itemize}
\item {} 
Pending (order still not submitted),

\item {} 
Submitted (order is submitted),

\item {} 
Cancelled (order is cancelled),

\item {} 
Processing (order is en route),

\item {} 
Delivered (order is delivered),

\item {} 
Error (order went wrong)

\end{itemize}


\chapter{Copy Data from HTTP}
\label{\detokenize{copy_data_HTTP:copy-data-from-http}}\label{\detokenize{copy_data_HTTP::doc}}
\sphinxhref{./images/copy\_data\_HTTP\_1.png}{\sphinxincludegraphics{{copy_data_HTTP_1}.png}}


\section{1. Go to \sphinxstylestrong{Order Status} select your delivered order and \sphinxstylestrong{Open Online Delivery Page}.}
\label{\detokenize{copy_data_HTTP:go-to-order-status-select-your-delivered-order-and-open-online-delivery-page}}
\sphinxhref{./images/copy\_data\_HTTP\_2.png}{\sphinxincludegraphics{{copy_data_HTTP_2}.png}}


\section{2. Download file(s) from the delivery page using \sphinxstylestrong{Right Click + “Save file under…”}}
\label{\detokenize{copy_data_HTTP:download-file-s-from-the-delivery-page-using-right-click-save-file-under}}
Please check, if the downloaded file is as big as given in the delivery
page and that the archive contains as many slots as intended. If not, download again.


\section{3. Move files from your local system to the \sphinxstyleemphasis{talos} server:}
\label{\detokenize{copy_data_HTTP:move-files-from-your-local-system-to-the-talos-server}}
\begin{sphinxVerbatim}[commandchars=\\\{\}]
cd Downloads

mv 1234567\PYGZhy{}1of1.tar /vols/talos/home/group\PYGZbs{}\PYGZus{}share/misc\PYGZbs{}\PYGZus{}documents/sat\PYGZbs{}\PYGZus{}archive\PYGZbs{}\PYGZus{}filling
\end{sphinxVerbatim}


\section{4. Log onto server altair as user “sat\_data” with:}
\label{\detokenize{copy_data_HTTP:log-onto-server-altair-as-user-sat-data-with}}
\begin{sphinxVerbatim}[commandchars=\\\{\}]
ssh \PYGZhy{}X sat\PYGZbs{}\PYGZus{}data\PYGZbs{}@altair
password: **Please ask**
\end{sphinxVerbatim}


\section{5. Move the file from talos to altair:}
\label{\detokenize{copy_data_HTTP:move-the-file-from-talos-to-altair}}
\begin{sphinxVerbatim}[commandchars=\\\{\}]
cd /vols/altair/datasets/eumcst/incoming/umarf/http/(year)

mv /vols/talos/home/group\PYGZbs{}\PYGZus{}share/misc\PYGZbs{}\PYGZus{}documents/sat\PYGZbs{}\PYGZus{}archive\PYGZbs{}\PYGZus{}filling/\PYGZbs{}*.tar /vols/altair/datasets/eumcst/incoming/umarf/http/(year)
\end{sphinxVerbatim}


\section{6. Extract your order:}
\label{\detokenize{copy_data_HTTP:extract-your-order}}
\begin{sphinxVerbatim}[commandchars=\\\{\}]
tar \PYGZhy{}xvf 1234567\PYGZhy{}1of1.tar
\end{sphinxVerbatim}


\chapter{Extract Data from the Tapes}
\label{\detokenize{extract_tapes:extract-data-from-the-tapes}}\label{\detokenize{extract_tapes::doc}}

\section{1. First you will have to check if the tape machine is ready.}
\label{\detokenize{extract_tapes:first-you-will-have-to-check-if-the-tape-machine-is-ready}}
\(\rightarrow\) Log into the LTO computer with:

\begin{sphinxVerbatim}[commandchars=\\\{\}]
ssh sat\PYGZbs{}\PYGZus{}data\PYGZbs{}@lto5.tropos.de
(password: **AbS11!**)
\end{sphinxVerbatim}

\(\rightarrow\) check the status with:

\begin{sphinxVerbatim}[commandchars=\\\{\}]
mtx \PYGZhy{}f /dev/sg1 status
\end{sphinxVerbatim}

\(\rightarrow\) The data transfer element has to be
empty.

\sphinxhref{./images/extract\_tapes\_1.png}{\sphinxincludegraphics{{extract_tapes_1}.png}}

If this isn’t the case you have to unload it with:

\begin{sphinxVerbatim}[commandchars=\\\{\}]
mtx \PYGZhy{}f /dev/sg1 unload
\end{sphinxVerbatim}


\section{2. Ask Hartwig for the key to access the server\sphinxhyphen{}room in building}
\label{\detokenize{extract_tapes:ask-hartwig-for-the-key-to-access-the-server-room-in-building}}
23.5.


\section{3. Use room \sphinxstylestrong{0.03} in the ground floor of 23.5 to access room}
\label{\detokenize{extract_tapes:use-room-0-03-in-the-ground-floor-of-23-5-to-access-room}}
\sphinxstylestrong{0.07}. Here you will find the rack with the tape\sphinxhyphen{}computer and
tape\sphinxhyphen{}drive right next to the door.


\section{4. At the tape\sphinxhyphen{}computer use the Next (►) or Previous key (◄) to}
\label{\detokenize{extract_tapes:at-the-tape-computer-use-the-next-or-previous-key-to}}
navigate to “Operations”. Use Enter to get to the selection of which of
the two magazines (each holding up to 4 tapes) you like to open.

\sphinxhref{./images/extract\_tapes\_2.png}{\sphinxincludegraphics{{extract_tapes_2}.png}}

Either use Enter again on “Unlock Left Magazine” or “Unlock Right
Magazine” and wait a brief moment.


\section{5. Kindly remove the unlocked magazine and insert or exchange}
\label{\detokenize{extract_tapes:kindly-remove-the-unlocked-magazine-and-insert-or-exchange}}
tape(s). Always start with \#1 (nearest position in the left magazine).

\sphinxhref{./images/extract\_tapes\_3.png}{\sphinxincludegraphics{{extract_tapes_3}.png}}

\sphinxstylestrong{Note} The left magazine contains:  \sphinxstyleemphasis{Positions 1} to \sphinxstyleemphasis{4} (starting with \#1)

and the right magazine contains:  \sphinxstyleemphasis{Position 5} to \sphinxstyleemphasis{8} (starting with \#5)


\section{6. Re\sphinxhyphen{}insert the magazine and wait for the tape\sphinxhyphen{}computer to read the}
\label{\detokenize{extract_tapes:re-insert-the-magazine-and-wait-for-the-tape-computer-to-read-the}}
tape.


\section{7. Turn off the light, leave the room and lock the door.}
\label{\detokenize{extract_tapes:turn-off-the-light-leave-the-room-and-lock-the-door}}

\section{8. Give the key back to Hartwig.}
\label{\detokenize{extract_tapes:give-the-key-back-to-hartwig}}

\section{9. Log into the LTO computer again}
\label{\detokenize{extract_tapes:log-into-the-lto-computer-again}}
\sphinxhref{./images/extract\_tapes\_4.png}{\sphinxincludegraphics{{extract_tapes_4}.png}}


\section{10. Go to the directory for tapes with:}
\label{\detokenize{extract_tapes:go-to-the-directory-for-tapes-with}}
\begin{sphinxVerbatim}[commandchars=\\\{\}]
cd /vols/talos/datasets/eumcst/incoming/umarf/tapes
\end{sphinxVerbatim}

Execute the shell\sphinxhyphen{}script \sphinxstylestrong{extract\_tape} for a single tape in \sphinxstyleemphasis{Positio 1}:

\begin{sphinxVerbatim}[commandchars=\\\{\}]
nohup extract\PYGZbs{}\PYGZus{}tape.sh \PYGZbs{}\PYGZdl{}TAPENAME \PYGZam{}
\end{sphinxVerbatim}

or the shell script \sphinxstylestrong{extract\_tapes} for multiple tapes:

\begin{sphinxVerbatim}[commandchars=\\\{\}]
nohup extract\PYGZbs{}\PYGZus{}tapes.sh \PYGZbs{}\PYGZdl{}TAPE1NAME \PYGZbs{}\PYGZdl{}TAPE2NAME \PYGZbs{}\PYGZdl{}TAPE3NAME \PYGZam{}
\end{sphinxVerbatim}

For the tape names you can simply use \sphinxcode{\sphinxupquote{mtx \sphinxhyphen{}f /dev/sg1 status}} again, it
shows them after VolumeTag=, remember to follow the position order.

\sphinxstylestrong{Note}: nohup lets you run programs even if you log out and writes any
messages into the file nohup.out.


\section{11. Now the data for Step III should be under:}
\label{\detokenize{extract_tapes:now-the-data-for-step-iii-should-be-under}}
\begin{sphinxVerbatim}[commandchars=\\\{\}]
/vols/talos/datasets/eumcst/incoming/umarf/tapes/(tapename)
\end{sphinxVerbatim}


\section{12. Unlike the http files the tape files don’t need to be unpacked,}
\label{\detokenize{extract_tapes:unlike-the-http-files-the-tape-files-don-t-need-to-be-unpacked}}
simply add them to the target folder:

\begin{sphinxVerbatim}[commandchars=\\\{\}]
/vols/altair/datasets/eumcst/incoming/umarf/http/(year)
\end{sphinxVerbatim}

\sphinxstylestrong{Note:} It can be useful to look into linux tape management to
understand the process and solve possibly occurring errors.

A starting point is:

\sphinxurl{https://www.cyberciti.biz/hardware/unix-linux-basic-tape-management-commands/}


\chapter{Filling gaps in TROPOS archive after delivery}
\label{\detokenize{filling_gaps:filling-gaps-in-tropos-archive-after-delivery}}\label{\detokenize{filling_gaps::doc}}
\sphinxstylestrong{Note:} the actual archive is at

\sphinxcode{\sphinxupquote{/vols/altair/datasets/eumcst/msevi\textbackslash{}\_pzs/l15.hrit/(year)}}

and

\sphinxcode{\sphinxupquote{/vols/altair/datasets/eumcst/msevi\textbackslash{}\_rss/l15.hrit/(year)}}

The filling process assumes that you have copied and extracted data from
tape or HTTP into a unique directory under:

\sphinxcode{\sphinxupquote{/vols/altair/datasets/eumcst/incoming/umarf/http/(year)}}

Now the following tasks need to be completed:
\begin{enumerate}
\sphinxsetlistlabels{\arabic}{enumi}{enumii}{}{.}%
\item {} 
Add data to archive

\item {} 
Update segment masks

\item {} 
Re\sphinxhyphen{}create gap info

\end{enumerate}

For each of these steps, you need to log onto server altair (see Step 4
of II.a)

In addition, you need to set the path to include the Anaconda python
environment by using the following command:

\sphinxcode{\sphinxupquote{export PATH=/vols/talos/local/anaconda/bin:\textbackslash{}\$PATH}}

\sphinxstylestrong{Note:} GNU Screen

As the steps take significant time, it is convenient to run sessions in
GNU Screen, so you can log out of the computer and later resume
sessions. You may look up
\sphinxhref{http://nathan.chantrell.net/linux/an-introduction-to-screen/}{*http://nathan.chantrell.net/linux/an\sphinxhyphen{}introduction\sphinxhyphen{}to\sphinxhyphen{}screen/*} for an introduction on GNU Screen.


\section{1. Adding data to the archive}
\label{\detokenize{filling_gaps:adding-data-to-the-archive}}
\sphinxcode{\sphinxupquote{fmcast\textbackslash{}\_ms15\textbackslash{}\_update.py \textbackslash{}\$DIR1 \textbackslash{}... \textbackslash{}\$DIRN \&}}

Example:

\sphinxcode{\sphinxupquote{cd /vols/altair/datasets/eumcst/incoming/umarf/http/2016}}

\sphinxcode{\sphinxupquote{nohup fmcast\textbackslash{}\_ms15\textbackslash{}\_update.py /vols/altair/datasets/eumcst/incoming/umarf/http/2016/\textbackslash{}* \&}}

This command traverses each directory tree. For each tarfile containing
HRITs, it checks whether a corresponding tarfile is already in the
archive, and whether it is complete. If it is incomplete, the missing
HRIT files are added otherwise a new tarfile is created in the archive.

Afterwards you will have to clear the directory:

\sphinxcode{\sphinxupquote{rm \textbackslash{}*.tar}}


\section{2. Updating the segment masks}
\label{\detokenize{filling_gaps:updating-the-segment-masks}}
This doesn’t have to be done every time. It is enough to do it after a
significant chunk of data is added.

Issue the command to update the segment mask e.g. for the year 2013 and
the Rapid\sphinxhyphen{}Scan\sphinxhyphen{}Service (‘rss’, for Prime\sphinxhyphen{}Service use ‘pzs’):

\sphinxcode{\sphinxupquote{fmcast\textbackslash{}\_ms15\textbackslash{}\_segmask.py \sphinxhyphen{}A \sphinxhyphen{}y 2013 \sphinxhyphen{}s \textbackslash{}\textquotesingle{}rss\textbackslash{}\textquotesingle{}}}

You can also specify the date interval for a month (here: Jan 2013):

\sphinxcode{\sphinxupquote{fmcast\textbackslash{}\_ms15\textbackslash{}\_segmask.py \sphinxhyphen{}A \sphinxhyphen{}m 2013\sphinxhyphen{}01 \sphinxhyphen{}s \textbackslash{}\textquotesingle{}rss\textbackslash{}\textquotesingle{}}}

A specific date (here: 1\textasciicircum{}st\textasciicircum{} Jan 2013):

\sphinxcode{\sphinxupquote{fmcast\textbackslash{}\_ms15\textbackslash{}\_segmask.py \sphinxhyphen{}A \sphinxhyphen{}d 2013\sphinxhyphen{}01\sphinxhyphen{}01 \sphinxhyphen{}s \textbackslash{}\textquotesingle{}rss\textbackslash{}\textquotesingle{}}}

A date period (equivalent to \sphinxhyphen{}m 2013\sphinxhyphen{}01):

\sphinxcode{\sphinxupquote{fmcast\textbackslash{}\_ms15\textbackslash{}\_segmask.py \sphinxhyphen{}A \sphinxhyphen{}d 2013\sphinxhyphen{}01\sphinxhyphen{}01,2013\sphinxhyphen{}02\sphinxhyphen{}01 \sphinxhyphen{}s \textbackslash{}\textquotesingle{}rss\textbackslash{}\textquotesingle{}}}

\sphinxstylestrong{Note:} this step is quite time\sphinxhyphen{}consuming, as it obtains the list of
files from each tarfile. Using nohup is recommended.


\section{3. Obtaining statistics/gap information**}
\label{\detokenize{filling_gaps:obtaining-statistics-gap-information}}
To find gaps for a year, do the following:

\sphinxcode{\sphinxupquote{cat /vols/altair/datasets/eumcst/msevi\textbackslash{}\_rss/meta/segmasks/2013/??/\textbackslash{}*.segmask \textbackslash{}| sort \textbackslash{}| \textbackslash{}\textbackslash{} fmcast\textbackslash{}\_ms15\textbackslash{}\_gaps.py \sphinxhyphen{}y 2013 \textbackslash{}\textgreater{} gaps\sphinxhyphen{}2013\sphinxhyphen{}rss.txt}}

This will produce output as follows:

\begin{sphinxVerbatim}[commandchars=\\\{\}]
2014\PYGZhy{}01\PYGZhy{}01 10:20 2014\PYGZhy{}01\PYGZhy{}02 10:45 rss 294

2014\PYGZhy{}01\PYGZhy{}14 09:00 2014\PYGZhy{}02\PYGZhy{}13 09:00 rss 8641
\end{sphinxVerbatim}

To split the file in large gaps, use the following command:

\sphinxcode{\sphinxupquote{cat gaps\sphinxhyphen{}2013\sphinxhyphen{}rss.txt \textbackslash{}| awk \textbackslash{}\textquotesingle{}\{if(\textbackslash{}\$6\textbackslash{}\textgreater{}=10)print\}\textbackslash{}\textquotesingle{} \textbackslash{}| less \textbackslash{}\textgreater{} gaps\sphinxhyphen{}2013\sphinxhyphen{}rss\sphinxhyphen{}long.txt}}

Small gaps can be viewed with:

\sphinxcode{\sphinxupquote{cat gaps\sphinxhyphen{}2013\sphinxhyphen{}rss.txt \textbackslash{}| awk \textbackslash{}\textquotesingle{}\{if(\textbackslash{}\$6\textbackslash{}\textless{}10)print\}\textbackslash{}\textquotesingle{} \textbackslash{}| less \textbackslash{}\textgreater{} gaps\sphinxhyphen{}2013\sphinxhyphen{}rss\sphinxhyphen{}short.txt}}

To create PDFs to print out and put in the sat\sphinxhyphen{}archiving file, use:

\sphinxcode{\sphinxupquote{enscript \sphinxhyphen{}p gaps\sphinxhyphen{}2013\sphinxhyphen{}pzs\sphinxhyphen{}long.ps gaps\sphinxhyphen{}2013\sphinxhyphen{}pzs\sphinxhyphen{}long.txt}}

ps2pdf gaps\sphinxhyphen{}2013\sphinxhyphen{}pzs\sphinxhyphen{}long.ps gaps\sphinxhyphen{}2013\sphinxhyphen{}pzs\sphinxhyphen{}long.pdf

Afterwards you can remove the ps\sphinxhyphen{}files

It’s also possible to obtain summary statistics for a year:

\sphinxcode{\sphinxupquote{cat /vols/altair/datasets/eumcst/msevi\textbackslash{}\_rss/meta/segmasks/2013/??/\textbackslash{}*.segmask \textbackslash{}| sort \textbackslash{}| \textbackslash{}\textbackslash{} fmcast\textbackslash{}\_ms15\textbackslash{}\_stats.py \sphinxhyphen{}y 2013}}

This will produce output as follows:

\begin{sphinxVerbatim}[commandchars=\\\{\}]
\PYGZbs{}\PYGZsh{}\PYGZbs{}\PYGZsh{}\PYGZbs{}\PYGZsh{}\PYGZbs{}\PYGZsh{}\PYGZbs{}\PYGZsh{}\PYGZbs{}\PYGZsh{}\PYGZbs{}\PYGZsh{}\PYGZbs{}\PYGZsh{}\PYGZbs{}\PYGZsh{}\PYGZbs{}\PYGZsh{}\PYGZbs{}\PYGZsh{}\PYGZbs{}\PYGZsh{}\PYGZbs{}\PYGZsh{}\PYGZbs{}\PYGZsh{}\PYGZbs{}\PYGZsh{}\PYGZbs{}\PYGZsh{}\PYGZbs{}\PYGZsh{}\PYGZbs{}\PYGZsh{}\PYGZbs{}\PYGZsh{}\PYGZbs{}\PYGZsh{}\PYGZbs{}\PYGZsh{}\PYGZbs{}\PYGZsh{}\PYGZbs{}\PYGZsh{}\PYGZbs{}\PYGZsh{}\PYGZbs{}\PYGZsh{}\PYGZbs{}\PYGZsh{}\PYGZbs{}\PYGZsh{}\PYGZbs{}\PYGZsh{}

\PYGZbs{}\PYGZsh{} Yearly reception statistics: 2013

\PYGZbs{}\PYGZsh{} Date \PYGZbs{}\PYGZsh{}Files \PYGZbs{}[\PYGZpc{}\PYGZbs{}] \PYGZbs{}\PYGZsh{}Slots \PYGZbs{}[\PYGZpc{}\PYGZbs{}] \PYGZbs{}\PYGZsh{}Compl. \PYGZbs{}[\PYGZpc{}\PYGZbs{}]

\PYGZbs{}\PYGZsh{}\PYGZbs{}\PYGZsh{}\PYGZbs{}\PYGZsh{}\PYGZbs{}\PYGZsh{}\PYGZbs{}\PYGZsh{}\PYGZbs{}\PYGZsh{}\PYGZbs{}\PYGZsh{}\PYGZbs{}\PYGZsh{}\PYGZbs{}\PYGZsh{}\PYGZbs{}\PYGZsh{}\PYGZbs{}\PYGZsh{}\PYGZbs{}\PYGZsh{}\PYGZbs{}\PYGZsh{}\PYGZbs{}\PYGZsh{}\PYGZbs{}\PYGZsh{}\PYGZbs{}\PYGZsh{}\PYGZbs{}\PYGZsh{}\PYGZbs{}\PYGZsh{}\PYGZbs{}\PYGZsh{}\PYGZbs{}\PYGZsh{}\PYGZbs{}\PYGZsh{}\PYGZbs{}\PYGZsh{}\PYGZbs{}\PYGZsh{}\PYGZbs{}\PYGZsh{}\PYGZbs{}\PYGZsh{}\PYGZbs{}\PYGZsh{}\PYGZbs{}\PYGZsh{}\PYGZbs{}\PYGZsh{}
2013\PYGZhy{}01\PYGZhy{}01 12672 100.00 288 100.00 288 100.00

\PYGZbs{}...

2013\PYGZhy{}12\PYGZhy{}31 12660 99.91 288 100.00 284 98.61

\PYGZbs{}\PYGZsh{}\PYGZbs{}\PYGZsh{}\PYGZbs{}\PYGZsh{}\PYGZbs{}\PYGZsh{}\PYGZbs{}\PYGZsh{}\PYGZbs{}\PYGZsh{}\PYGZbs{}\PYGZsh{}\PYGZbs{}\PYGZsh{}\PYGZbs{}\PYGZsh{}\PYGZbs{}\PYGZsh{}\PYGZbs{}\PYGZsh{}\PYGZbs{}\PYGZsh{}\PYGZbs{}\PYGZsh{}\PYGZbs{}\PYGZsh{}\PYGZbs{}\PYGZsh{}\PYGZbs{}\PYGZsh{}\PYGZbs{}\PYGZsh{}\PYGZbs{}\PYGZsh{}\PYGZbs{}\PYGZsh{}\PYGZbs{}\PYGZsh{}\PYGZbs{}\PYGZsh{}\PYGZbs{}\PYGZsh{}\PYGZbs{}\PYGZsh{}\PYGZbs{}\PYGZsh{}\PYGZbs{}\PYGZsh{}\PYGZbs{}\PYGZsh{}\PYGZbs{}\PYGZsh{}\PYGZbs{}\PYGZsh{}

Combined 3951734 85.44 89859 85.48 89375 85.02

\PYGZbs{}\PYGZsh{}\PYGZbs{}\PYGZsh{}\PYGZbs{}\PYGZsh{}\PYGZbs{}\PYGZsh{}\PYGZbs{}\PYGZsh{}\PYGZbs{}\PYGZsh{}\PYGZbs{}\PYGZsh{}\PYGZbs{}\PYGZsh{}\PYGZbs{}\PYGZsh{}\PYGZbs{}\PYGZsh{}\PYGZbs{}\PYGZsh{}\PYGZbs{}\PYGZsh{}\PYGZbs{}\PYGZsh{}\PYGZbs{}\PYGZsh{}\PYGZbs{}\PYGZsh{}\PYGZbs{}\PYGZsh{}\PYGZbs{}\PYGZsh{}\PYGZbs{}\PYGZsh{}\PYGZbs{}\PYGZsh{}\PYGZbs{}\PYGZsh{}\PYGZbs{}\PYGZsh{}\PYGZbs{}\PYGZsh{}\PYGZbs{}\PYGZsh{}\PYGZbs{}\PYGZsh{}\PYGZbs{}\PYGZsh{}\PYGZbs{}\PYGZsh{}\PYGZbs{}\PYGZsh{}\PYGZbs{}\PYGZsh{}
\end{sphinxVerbatim}


\chapter{Indices and tables}
\label{\detokenize{index:indices-and-tables}}\begin{itemize}
\item {} 
\DUrole{xref,std,std-ref}{genindex}

\item {} 
\DUrole{xref,std,std-ref}{modindex}

\item {} 
\DUrole{xref,std,std-ref}{search}

\end{itemize}



\renewcommand{\indexname}{Index}
\printindex
\end{document}